\documentclass[a4paper, 11pt]{article}
\begin{document}
\title{Report: Torque in a Variable Reluctance Machine}
\author{Barış Kuseyri}
\date{8 March 2020}
\maketitle

\pagenumbering{roman}
\tableofcontents
\newpage

\section{Introduction}
\label{sec1}
\subsection{Aims}

This report examines a basic variable reluctance machine. A set of values, dimensions vsvs are given. From these values an analytical model of the machine is obtained. This model includes an analytical formula for the relucatance and the inductance of the system as a fucntion of rotation of the variable reluctance machine. Then, torque characteristics are plotted, while machine coil is under DC excitation.
\subsection{Content}
The report proceeds with a methodology section discussing the methods used throughout this project. Then, specifics of the methods are presented in the subsequent section. The results are stated after methodology. The results are assessed in the Evaluation section. The report ends with conclusion and final remarks.

\section{Method}
This project examines the variable reluctance machine by carrying out analytical and FEA modelling. The analytical model is achieved by implementing a series of equations, complying with the theory. 
\section{Modelling}
\subsection{Analytical Modelling}
\subsection{FEA Modelling: 2D - Linear Materials}

\subsection{FEA Modelling: 2D - Nonlinear Materials}
\subsection{Control Method}

\begin{equation}
\label{skineffect}
	T=\frac{1}{2}i^2\frac{d\mathrm L(\theta)}{d\mathrm (\theta)}
    \label{biotsavart}
\end{equation}
\begin{equation}
\label{skineffect}
	\mathrm{i(t)}=\mathrm{I_m}*\mathrm{sin(w_rt)}
\end{equation}
\begin{equation}
    \delta=\sqrt{\frac{2\rho}{\omega\mu}}
\end{equation}

\subsection{Analytical Modelling}
\subsection{Motion Animation}
\subsection{FEA Modelling: 3D}
\section{Results}
\section{Evaluation}

\end{document}